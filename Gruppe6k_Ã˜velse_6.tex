%article using font-size 12 and A4-size
\documentclass[a4paper,12pt]{article}

%use danish hyphenation and titles
%handle utf8-characters
\usepackage[danish]{babel}
\usepackage[utf8]{inputenc}
\usepackage[T1]{fontenc}

%for images
\usepackage[pdftex]{graphicx}

%allow nested figures
\usepackage{subfigure}

%control line spacing
\usepackage{setspace}
%\singlespacing
\onehalfspacing
%\doublespacing

%set margins
%\usepackage[margin=0.75in]{geometry}

%allows margin-notes
%good for work in progress papers
%\usepackage{marginnote}

%allows pretty quoting using ``'' or `'
\usepackage{upquote}

%two definitions of the color grey
\usepackage{color}
\definecolor{listinggray}{gray}{0.9}
%\definecolor{lbcolor}{rgb}{0.9,0.9,0.9}

%allows pretty source code
\usepackage{listings}
\lstset{
	language=,
	literate=
		{æ}{{\ae}}1
		{‚àö‚àè}{{\o}}1
		{å}{{\aa}}1
		{‚àö√ú}{{\AE}}1
		{√ò}{{\O}}1
		{√Ö}{{\AA}}1,
	backgroundcolor=\color{listinggray},
	tabsize=3,
	rulecolor=,
	basicstyle=\scriptsize,
	upquote=true,
	aboveskip={1.5\baselineskip},
	columns=fixed,
	showstringspaces=false,
	extendedchars=true,
	breaklines=true,
	prebreak =\raisebox{0ex}[0ex][0ex]{\ensuremath{\hookleftarrow}},
	frame=single,
	showtabs=false,
	showspaces=false,
	showstringspaces=false,
	identifierstyle=\ttfamily,
	keywordstyle=\color[rgb]{0,0,1},
	commentstyle=\color[rgb]{0.133,0.545,0.133},
	stringstyle=\color[rgb]{0.627,0.126,0.941},
}
%captions on listings
\usepackage[center,font=small,labelfont=bf,textfont=it]{caption}

%allows fancy enumeration
\usepackage{enumerate}

%allows use of the BibTex for the bibliography
\usepackage[numbers]{natbib}

%make references and URLs in the pdf to clickable links
\usepackage{hyperref}

%proper header formatting
\usepackage{fancyhdr}
\pagestyle{fancy}
\lhead[]{} %clear standard settings
\chead[]{} %clear standard settings
\rhead[]{\rightmark} %current section
\lfoot[]{} %clear standard settings
\cfoot[]{\thepage} %current page number 
\rfoot[]{} %clear standard settings

%define paper title, author and date
%if date is empty, defaults to current date
\title{Kontekstuel analyse}
\author{}
\date{Oktober 9th - 2013}

\begin{document}
\maketitle %insert the defined title
\tableofcontents %generate table of content
\newpage

\clearpage %clears float-buffer - inserting those missing - and starts new page

\section{PACT analyse}
\subsection{People}
Disse brugere vil man kunne forvente vil bruge låsesystemet.

\begin{itemize}
    \item Voksne der er husejere
    \item Børn til husejere
    \item Personer der er åben for ny teknologi
    \item Personer der er erfarne it-brugere.
    \item Typer, der gerne vil holde øje med deres hjem.
\end{itemize}

\subsection{Activities}
Dette er opgaverne man vil kunne udføre ved låsesystemet.

\begin{itemize}
    \item Låse op og i
    \item Aflæse om låsen er åben eller låst
    \item Give en midlertidig nøgle
    \item Gøre en nøgle ugyldig, så den ikke kan bruges til låsen
    \item Se en log over interaktioner med låsen på en bestemt dato
    \item Ændre brugerprivilegier eks. lade en bruger uddele nøgler eller se en log over interaktioner med låsen på en bestemt dato
    \item Se hvem der har nøgle til låsen
\end{itemize}


\subsection{Context}
Dette er situationer brugere muligvis vil bruge låsesystemet.

\begin{itemize}
    \item En person skal lukke en håndværker ind til reparationsarbejde mens han
ikke er hjemme.
    \item Når du får et opkald fra en i familien som har glemt sin nøgle og vil ind.
    \item En person er taget på arbejde og kan ikke huske om vedkommende låste døren.
    \item Brugeren vil finde ud af hvem der har brugt låsen den forrige dag.
    \item Brugeren vil fratage en lejers digitale nøgle.
    \item Hvis du fysisk ikke kan bevæge dig hen til døren.
\end{itemize}

\subsection{Technology}
Application vil blive kørt på en smartphone, tablet eller datamat med adgang til internettet.

\section{Tjekliste}
Følgende tjekliste bliver brugt som støtte i interviewene til at guide samtalen. \\
Disse punkter indleder interviewet og forklarer brugeren hvad der skal foregå og hvorfor:

\begin{itemize}
    \item Vi vil vise dig en tidlig udgave (prototype) af låsesystemet.
    \item Hvad er dine internet-færdigheder? Bruger du din smartphone ofte? Hvilke slags app bruger du?
\end{itemize}

Disse punkter sikrer at relevante emner omkring brugerens forventninger til låsesystemet bliver dækket i interviewene:

\begin{itemize}
    \item Hvor ofte bruger du en lås?
    \item Hvor mange nøgler er der i dit nøglebundt?
    \item Har du mange nøgler til få steder, eller få nøgler til mange steder?
    \item Bruger du i forvejen nogle internetstyrede løsninger i dit hjem (fx. lys, musikanlæg etc.)?
    \item Hvornår smed du sidst din nøgler væk?
    \item Hvornår skulle du sidst bruge en kopi af en nøgle?
    \item Har du brugt en digital lås før? (Evt. hotel el. lign.) ?
    \item Vil du betegne dig selv som en person der generelt har tillid til ny teknologi?
    \item Hvilke forskellige typer af nøgler kender du (kort, chip, fingeraftryk etc.)?
    \item Hvad syntes du om at kunne åbne og låse med din smartphone?
    \item Hvilke funktioner kunne låsesystemet have som du syntes var smart?
    \item Hvilke funktioner kunne låsesystemet have som du syntes var unødvendigt?
    \item Kunne du finde på at bruge låsesystemet?
\end{itemize}
Følgende punkter vil blive stillet efter demonstrationen af prototypen:

\begin{itemize}
    \item Hvordan lever låsesystemet op til dine forventninger?
    \item Syntes du der mangler noget?
    \item Kan du nævne to ting du kunne lide og to ting du ikke kunne lide ved låsesystemet?
    \item Ville du vælge en sådan løsning til dit eget hjem?
    \item Hvilke sider syntes du kunne blive forbedret?
    \item Hvorfor ville du eller ville du ikke bruge sådan et låsesystem i fremtiden?
    \item Ville du anvende dette låsesystem som supplement til fysiske nøgler, eller som din primære låsemetode?
\end{itemize}


\end{document}